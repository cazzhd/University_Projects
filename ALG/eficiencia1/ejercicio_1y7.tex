\documentclass{article}
\usepackage[english]{babel}
\usepackage[utf8]{inputenc}
\usepackage{fancyhdr}
\usepackage{hyperref}
\usepackage{mathtools}
\usepackage[table]{xcolor}
\usepackage{amsmath,amssymb}
\usepackage{listings}
\pagestyle{fancy}

%TAG LEFT RIGHT EQUATION%
\usepackage{amsmath}
\makeatletter
\newcommand{\leqnomode}{\tagsleft@true\let\veqno\@@leqno}
\newcommand{\reqnomode}{\tagsleft@false\let\veqno\@@eqno}
\makeatother

\fancyhf{}
\rhead{Lukas Häring García - 2ºC}
\lhead{Relación tema 1. Eficiencia}
\rfoot{Página \thepage}
\begin{document}
	\begin{enumerate}
		\item De las siguientes afirmaciones, indicar cuales son ciertas y cuales no:
		\begin{center}
			\fbox{\begin{minipage}{9cm}
				$$f(n) \in O(g(n)) \xLeftrightarrow{(1)} \displaystyle{\lim_{n \to +\infty}}  \frac{f(n)}{g(n)} = 0 \xLeftrightarrow{(2)} g(n) \in \Omega(f(n))) $$
			\end{minipage}}
		\end{center}
		1. Es la resolución de "BigO Notation", 2. Cálculo de "Omega Notation".
		\begin{enumerate}
			\item $n^2\in (O\wedge\Omega)(n^3)$\\*
				1. $\displaystyle{\lim_{n \to +\infty}} \frac{n^2}{n^3} = \displaystyle{\lim_{n \to +\infty}} \frac{1}{n} = 0$\\*
				2. Este no puede ser, pues ya es O(n)
			\item $n^3\in (O\wedge\Omega)(n^2)$\\*
				1. $\displaystyle{\lim_{n \to +\infty}} \frac{n^3}{n^2} = \displaystyle{\lim_{n \to +\infty}} n = +\infty$\\*
				2. Por la definición (2) y (a.1), podemos asegurar que $n^3\in \Omega(n^2)$.
			\item $2^{n+1}\in (O\wedge\Omega)(2^n)$\\*
				1. $\displaystyle{\lim_{n \to +\infty}} \frac{2^{n+1}}{2^n} = \displaystyle{\lim_{n \to +\infty}} \frac{2^{n}\cdot 2}{2^n} = \displaystyle{\lim_{n \to +\infty}} 2 = 2$\\*
				2. $2^{n+1}\ge 2^n \Leftrightarrow 2^{n}\cdot 2\ge 2^n \Leftrightarrow 2 \ge 1 \Rightarrow 2^{n+1}\in \Omega(2^{n})$.
			\item $(n+1)!\in (O\wedge\Omega)(n!)$\\*
				1. $\displaystyle{\lim_{n \to +\infty}} \frac{(n+1)!}{n!} = \displaystyle{\lim_{n \to +\infty}} \frac{(n+1)\cdot  n!}{n!} = \displaystyle{\lim_{n \to +\infty}} (n+1) = +\infty$\\*
				2. $(n+1)!\ge n! \Leftrightarrow (n+1)\cdot n! \ge n!\Leftrightarrow n \ge 0 \Rightarrow (n+1)!\in \Omega(n!)$.
			\item $f(n)\in (O\wedge\Omega)(n)\left(=\displaystyle{\lim_{n \to +\infty}}\dfrac{f(n)}{n}=0\right)\Rightarrow 2^{f(n)}\in (O\wedge\Omega)(2^n)$\\*
				1. $\displaystyle{\lim_{n \to +\infty}} \dfrac{2^f(n)}{2^n}=\displaystyle{\lim_{n \to +\infty}}2^{(f(n)-n)\cdot\dfrac{n}{n}}=\displaystyle{\lim_{n \to +\infty}}2^{(\dfrac{f(n)}{n}-1)n}=\displaystyle{\lim_{n \to +\infty}}2^{-n}$ \\*
				2. $\displaystyle{\lim_{n \to +\infty}}2^{(f(n)-n)\cdot\dfrac{f(n)}{f(n)}}=\displaystyle{\lim_{n \to +\infty}}2^{(1-\dfrac{n}{f(n)})\cdot f(n)}=\displaystyle{\lim_{n \to +\infty}}2^{-+\infty\cdot f(n)}$
			\item $3^n\in (O\wedge\Omega)(2^n)$\\*
				1.  $\displaystyle{\lim_{n \to +\infty}} \frac{3^{n}}{2^n} = \displaystyle{\lim_{n \to +\infty}} \left(\frac{3}{2}\right)^{n}=+\infty$, ya que $\frac{3}{2}$ es mayor que 1, diverge.\\*
				2. ¿$3^n\ge 2^n \Leftrightarrow log(3)\cdot n \ge log(2)\cdot n$? Lo que es cierto.
			\item $log(n)\in (O\wedge\Omega)(n^{1/2})$\\*
				1. $\displaystyle{\lim_{n \to +\infty}} \frac{log(n)}{n^{1/2}} = \frac{+\infty}{+\infty} = \text{ l'Hôpital } = \displaystyle{\lim_{n \to +\infty}} \frac{\frac{1}{n}}{\frac{1}{2\sqrt{n}}}=\displaystyle{\lim_{n \to +\infty}}\frac{2\sqrt{n}}{n}=0$\\*
				2. Este converge a 0, por lo que no puede ser $\Omega(n^1/2)$.
			\item $n^{1/2}\in (O\wedge\Omega)(log(n))$\\*
				1. Del ejercicio anterior, podemos asegurar que este no puede ser $O(log(n))$, puesto que antes no divergió.\\*
				2. Al igual que el apartado (b.2), podemos asegurar que sí lo es.
		\end{enumerate}
\newpage
		\item Demostrar que $\forall k\in N$, $log(n)^k\in O(n)$.\\*
		Utilizando la definición (1), es equivalente a resolver:
		\[
			\displaystyle{\lim_{n \to +\infty}} \frac{log(n)^k}{n}=\dfrac{+\infty}{+\infty}=\text{ l'Hôpital }=\displaystyle{\lim_{n \to +\infty}} \frac{klog(n)^{k-1}}{n}=\dfrac{+\infty}{+\infty}
		\]
		Repitiendo la regla de l'Hôpital, $k-1$ vez más, obtenemos:
		\[
			\displaystyle{\lim_{n \to +\infty}} \frac{k!\log(n)}{n}=\dfrac{+\infty}{+\infty}=\text{ l'Hôpital }=\displaystyle{\lim_{n \to +\infty}} \frac{k!}{n}=0.
		\]
		\item Dada la siguiente función.
\begin{lstlisting}[language=c++]
int Raro(const int* a, int i0,int i1){
	if (i0 >= i1) return a[i1];
	int m = (i0 + i1)/2;
	int t = (i1 - i0)/3;
	int r0 = Raro(a, i0, i0 + terc);
	int r1 = Raro(a, i1 - t, i1);
	return a[m] + r0 + r1;
}
\end{lstlisting}
\begin{enumerate}
	\item Calcular el tiempo de ejecucion de la funcion Raro(a,0,n-1) suponiendo que n es potencia de 3.\\*
	Definimos $n=i1-i0=3^k$ (1) según la función dada, analizando el código, obtenemos el siguiente esquema iterativo:\\*\\*
	$
	\overbrace{\phantom{This text will bi}}^\text{Raro(a, i0, i0 + terc)}
	\phantom{This text will biasdsdasddas2344sdadasa}
	\overbrace{\phantom{This text will bi}}^\text{Raro(a, i1 - t, i1)}
	$\\*
	\begin{tabular}{ | p{2cm} | p{2cm} | p{1cm} | p{2cm} | p{2cm} | }
		\hline
		\begin{center}
			$T\left(\dfrac{n}{3}\right)$
		\end{center} & \cellcolor{blue!25} & \begin{center}
		$a\left[\dfrac{n}{2}\right]$
	\end{center} & \cellcolor{blue!25} & \begin{center}
		$T\left(\dfrac{n}{3}\right)$
	\end{center} \\ \hline
	\end{tabular}
	La zona coloreada es aquella parte donde no se ejecuta la función.
	\[   
	T(n) = 
	\begin{cases}
	1 & n \le 0\\
	1 + 2T\left(\dfrac{n}{3}\right) & n \ge 1\\
	\end{cases}
	\]
	Realizando la substitución (1).
	\[
	T(3^k)=1+2T(3^{k-1})
	\]
	\begin{enumerate}
		\item La parte \textbf{homogénea} es: $T(3^k)-2T(3^{k-1})=x-2=0$.
		\item La parte \textbf{no homogénea} es: $1=1\cdot 1^k$.
	\end{enumerate}
	Obtenemos como raíces, $\alpha=2,1$, el polinomio resultante es:
	\[
		T(3^k)=2^kA+1^kB=T(n)=2^{\dfrac{\log(n)}{\log(3)}}A+B=n^{\dfrac{\log(2)}{\log(3)}}A+B=n^{0.630}A+B
	\]
	\item Dar una cota de complejidad para dicho tiempo de ejecución.\\*
	$\displaystyle{\lim_{n \to +\infty}} \frac{n^{0.630}}{n}=\displaystyle{\lim_{n \to +\infty}} \dfrac{1}{n^{0.369}}=0\Rightarrow O(T(n))=O(n^{0.630})\in O(n)$
\end{enumerate}
\newpage
\item Consideremos la siguiente función que obtiene de un árbol binario:

\begin{lstlisting}[language=c++]
int Altura(bintree<int>& ab,bintree<int>::nodo n){
	if (n.nulo()){
		return 0;
	}
	int h0 = Altura(ab, ab.hizq(n));
	int h1 = Altura(ab, ab.hdrcah(n));
	return 1 + max(h0, h1);
}
\end{lstlisting}
Definimos $n$ como la cantidad de nodos de nuestro sub-árbol binario y $k$ como la cantidad a la izquierda del mismo ($n-k-1$ la cantidad derecha).
\[   
T(n) = 
\begin{cases}
1 & n = 0\\
T(k)+T(n-k-1)+1 & n \ge 1\\
\end{cases}
\]
El peor caso ocurre cuando todos los nodos están \textbf{siempre} en un mismo lado, eso es equivalente a decir que $k=0$.
\[
	T(n)=T(0)+T(n-1)+1=1+T(n-1)+1=T(n-1)+2
\]
\[
	T(n-1)=T(n-2)+2\Rightarrow T(n)=T(n-2)+2+2
\]
Repitiendo $n$ veces la recurrencia, obtenemos la solución a la recurrencia.
\[
	O(T(n))=O(c_1+2c_2n)\in O(n)
\]
Ahora bien si a la izq. como a la der. hay la mitad del total $k=\dfrac{n=2^d}{2}$.
\[
T(n)=T\left(\dfrac{n}{2}\right)+T\left(\dfrac{n}{2}\right)+1=1+2T\left(\dfrac{n}{2}\right)\Rightarrow T(2^d)=1+2T(2^{d-1})
\]
\begin{enumerate}
	\item La parte \textbf{homogénea} es: $T(2^d)-2T(2^{d-1})=x-2=0$.
	\item La parte \textbf{no homogénea} es: $1=1\cdot 1^d$.
\end{enumerate}
Obtenemos como ecuación no recurrente:
\[
	T(n)=2^dc_1+1^dc_2=nc_1+c_2\Rightarrow \Omega(T(n))=\Omega(nc_1+c_2)\in \Omega(n)
\]
Como ambas cotas concuerdan.
\[\Omega(T(n))\subseteq O(T(n))\Rightarrow \Theta(T(n))\in\Theta(n)\]
\newpage
\item Ordenar las siguientes funciones de acuerdo a su velocidad de crecimiento:
\[
n,\sqrt{n}, \log(n), \log(\log(n)), \log^2(n),\dfrac{n}{log(n)},\sqrt{n}\log^2(n),\left(\dfrac{1}{3}\right)^n,\left(\dfrac{3}{2}\right)^n,17,n^2
\]
Ordenando según la velocidad de crecimiento, definimos el símbolo $<$ como una regla de orden decreciente entre el tiempo de ejecución de dos funciones, obtenemos:

\reqnomode
\begin{equation}
	17<\left(\dfrac{1}{3}\right)^n<\log(\log(n))<\log(n)<\log^2(n)<\sqrt{n}\tag{*}
\end{equation}

\leqnomode
\begin{equation}
	<\sqrt{n}\log^2(n)<\dfrac{n}{log(n)}<n<n^2<\left(\dfrac{3}{2}\right)^n\tag{*}
\end{equation}
\item Suponiendo que $T_1\in O(f)$ y que $T_2\in O(f)$, indicar cuáles de las siguientes afirmaciones son ciertas:
\[
	T_1, T_2\in O(f)\Rightarrow \displaystyle{\lim_{n \to +\infty}} \dfrac{T_1(n)}{f(n)}=\displaystyle{\lim_{n \to +\infty}}\dfrac{T_1(n)}{f(n)}=0
\]
\begin{enumerate}
	\item $T_1+T_2\in O(f)$
	\[
		\displaystyle{\lim_{n \to +\infty}} \dfrac{T_1(n)+T_2(n)}{f(n)}=\displaystyle{\lim_{n \to +\infty}} \dfrac{T_1(n)}{f(n)}+\displaystyle{\lim_{n \to +\infty}} \dfrac{T_2(n)}{f(n)}=0+0=0
	\]
	\item $T_1-T_2\in O(f)$
	\[
		\displaystyle{\lim_{n \to +\infty}} \dfrac{T_1(n)-T_2(n)}{f(n)}=\displaystyle{\lim_{n \to +\infty}} \dfrac{T_1(n)}{f(n)}-\displaystyle{\lim_{n \to +\infty}} \dfrac{T_2(n)}{f(n)}=0-0=0
	\]
	\item $\dfrac{T_1}{T_2}\in O(f)$
	\[
	\displaystyle{\lim_{n \to +\infty}} \dfrac{T_1(n)}{T_2(n)f(n)}=\displaystyle{\lim_{n \to +\infty}} \dfrac{1}{T_2(n)}\cdot\displaystyle{\lim_{n \to +\infty}} \dfrac{T_1(n)}{f(n)}=\displaystyle{\lim_{n \to +\infty}} \dfrac{1}{T_2(n)}\cdot 0=0
	\]
	\item $T_1\in O(T_2)$
	\[
		\displaystyle{\lim_{n \to +\infty}} \dfrac{T_1(n)}{T_2(n)}=\displaystyle{\lim_{n \to +\infty}} \dfrac{T_1(n)f(n)}{T_2(n)f(n)}=\displaystyle{\lim_{n \to +\infty}} \dfrac{T_1(n)}{f(n)}\cdot \displaystyle{\lim_{n \to +\infty}} \dfrac{f(n)}{T_2(n)}=0\cdot \displaystyle{\lim_{n \to +\infty}} \dfrac{f(n)}{T_2(n)}
	\]
	c) y d) se cumplen $\Leftrightarrow \displaystyle{\lim_{n \to +\infty}} T_2(n) \ne 0$
\end{enumerate}

\newpage 
		\item Resolver la ecuación:
		\[   
		T(n) = 
		\begin{cases}
		0 & n = 0\\
		\dfrac{1}{n}\left(\smashoperator[r]{\sum_{i=0}^{n-1}}T(i)\right)+cn & n \ge 1\\
		\end{cases}
		\]
		Para realizar los cálculos menos enrevesados, definimos:
		\[
			K_\alpha=\smashoperator[r]{\sum_{i=0}^{\alpha}}T(i)=T(\alpha)+\smashoperator[r]{\sum_{i=0}^{\alpha-1}}T(i)=T(\alpha)+K_{\alpha-1}\tag{*}
		\]
		Calculemos el término $T(n-1)$.
		\[
			T(n-1)=\dfrac{1}{n-1}K_{n-2}+c(n-1)\Rightarrow K_{n-2}=(T(n-1)-c(n-1))(n-1)
		\]
		Simplificando, obtenemos el valor de $K_{n-2}$.
		\[
			K_{n-2}=T(n-1)(n-1)-c(n-1)^2
		\]
		Ahora bien, vamos a aplicar la igualdad (*) para $T(n)$.
		\[
			T(n)=\dfrac{1}{n}K_{n-1}+cn=\dfrac{1}{n}(T(n-1)+K_{n-2})+cn
		\]
		Usando el valor de $K_{n-2}$ que hemos calculado antes.
		\[
			T(n)=\dfrac{1}{n}(T(n-1)+T(n-1)(n-1)-c(n-1)^2)+cn
		\]
		Agrupamos y expandamos los polinomios.
		\[
			T(n)=\dfrac{T(n-1)(n-1+1)}{n}-\left(\dfrac{-2n+1}{n}\right)c
		\]
		Simplificamos.
		\[
			T(n)=T(n-1)+2c-\dfrac{1}{n}c
		\]
		Para acortar un poco, vamos a llamar $T(n)=t_n$.\\* Como ya tenemos calculado $t_n$, vamos a calcular $t_{n-1}$.
		\[
			t_{n-1}=t_{n-2}+\left(2c-\dfrac{1}{n-1}c\right)\Rightarrow t_n=\left(t_{n-2}+2c-\dfrac{1}{n-1}c\right)+\left(2c-\dfrac{1}{n}c\right)
		\]
		Agrupamos los términos de $t_n$
		\[
			t_{n}=t_{n-2}+(2c + 2c)-\left(\dfrac{1}{n}+\dfrac{1}{n-1}\right)c
		\]
		Si seguimos expandiendo, este patrón será visible para cada iteración, por lo que la solución a la ecuación es
		\[
			t_n = 2nc - \left(\smashoperator[r]{\sum_{k=0}^{n}} \dfrac{1}{k}\right)c = 2nc - \mathcal{H}_n c
		\]
		Dónde $\mathcal{H}_n$ es conocida como la serie armónica hasta el término n-ésimo.
	\end{enumerate}
\end{document}